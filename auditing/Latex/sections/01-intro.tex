% !TEX root = ../main.tex
\section{Introductory Remarks}

In certain common circumstances, firms operating in the blockchain space will require their financial statements to be audited by an external firm. Annual audits are legally mandatory for publicly traded companies in most countries, and audits might also be required when a firm borrows from a bank or raises capital from investors. Auditing is a timely subject as, at the time of writing, major auditing firms are hesitating to provide certification to a wide range of businesses in the blockchain sector due to a perception of insurmountable business risk associated with these clients.  This creates friction for firms wanting to raise capital in traditional ways. 

When assessing whether or not to take on a new client, auditors lack experience in this sector and therefore are unable to develop expectations of financial performance as a way of challenging financial statement assumptions. Due to the complex and rapidly changing technological environment, auditors are also unable to keep pace with the changes and develop the in-depth knowledge of their clients{'}  businesses required for performing an audit. Finally, auditors are wary of accepting clients that hold a significant amount of cryptoassets as this space is largely unregulated. A lack of third-party oversight puts an onus on the auditor, further increasing their risk exposure.  

In this paper we explore why and provide a comprehensive overview of the challenges auditors perceive as novel, we form parallels to auditing approaches used today, and critically assess the extent to which these challenges are barriers. Altogether, we find that while this environment is new, the challenges presented are different incarnations of issues already addressed on traditional audit clients. Therefore, we conclude that many entities in this space are auditable, subject to certain caveats. 


\paragraph{\textbf{Methodology.}} To ensure a comprehensive overview of the changes facing auditors of companies deploying blockchain technologies, we first used structured brainstorming within our multidisciplinary research team, which includes expertise in both auditing and blockchain technologies. Once our preliminary list of challenges was established, we vetted our results through informal discussions with over a dozen individuals working in the field, including at the Big Four\footnote{The Big Four are the four biggest auditing firms in the world which includes, Ernst \& Young (EY), Deloitte \& Touche, KPMG and PricewaterhouseCoopers (PwC).} and mid-sized accounting firms, and used these interviews to augment our list. The contribution of this paper is to provide a comprehensive list of challenges, rather than determining the relative importance of each or any broader concepts across the industry---thus we did not find it necessary to apply qualitative data analysis (\eg grounded theory) to the interviews; instead, we simply extracted the challenges raised which we initially missed.


\paragraph{\textbf{Relevance.}} Our work can be viewed as a case study of cryptographers working with an outside profession for addressing a real-world problem. The result is not a new protocol but a two-way knowledge transfer between professions. While displacing auditors is occasionally a target of cryptographers~\cite{narula2018zkledger,dagher2015provisions}, auditing itself is also occasionally studied directly at venues like Financial Cryptography~\cite{grigg2000financial}.


% = = = = = = = = = = = = = = = = = = = = = = = = = = = = = = = = = = = = = = = = = =

\section{Preliminaries and Related Work}
Financial statements are prepared by a firm to summarize its assets and liabilities at year-end, the resulting changes in firm equity, and the firm{'}s revenue, expenses, and cash flows across the year. Because the firm generally wants to present the most optimistic view of its financial health, the firm may hire an auditor to ensure it is a realistic picture of the company{'}s financial performance. In doing so, auditors are expected to comply with Generally Accepted Auditing Standards (GAAS) and the Code of Conduct of their professional order~\cite{strother1975establishment}. In addition to an audited statement, the firm will produce quarterly statements that are reviewed by a third party auditor to a less rigorous standard.

%NOTE: Note that there are auditing and accounting papers in the bib/zotero that have not been cited here! {blaze1998compliance}
%------------------------------------------------------------------------------------------------
\subsection{Primary Stakeholders in an Audit}

\paragraph{Firm Management.} Financial statements are prepared by and remain the responsibility of firm management. Many assumptions and estimates are required to prepare this report. A firm's CEO and CFO must certify the adequacy and effectiveness of the firm{'}s internal controls over financial reporting, which poses a reputational risk for firm directors as they will be held accountable for financial statements that are misstated. Management will seek an audit to comply with regulatory requirements, or because they feel it will give them a competitive advantage over non-audited firms in raising capital and lowering their lending rates. In a market where auditors are reluctant or audits are expensive, firms can use their ability to obtain audited financial statements as a barrier against new entrants to the blockchain sector. 

\paragraph{Auditor.} The auditor for a given firm is selected by the firm itself. When an auditor provides an audit opinion, there is always a chance that fraud or misrepresentation went undetected in the statements. If this fraud is subsequently uncovered, the auditor could be liable for substantial penalties. For instance, Deloitte LLP paid a settlement of \$150M for failure to find fraud at mortgage broker Taylor, Bean \& Whitaker~\cite{FTDeloitte2018}.

\paragraph{Investors.} The primary users of financial statements are external investors who make decisions about whether or not to lend money or invest capital in a target firm. Investors use financial statements to assess their potential return on investment. Although \textit{caveat emptor} is an important consideration, investors rely on the existence of an auditor{'}s report as a signal of the reliability of financial disclosures.
 
\paragraph{Financial Regulators.} Regulators span different government agencies with a spectrum of concerns including financial fraud, taxation, anti-money laundering, and know your customer (KYC) rules. Financial audits are central to the role of security regulators, who require publicly traded firms to obtain an annual unqualified audit opinion on their financial statements for consumer protection. Failure to do so, would put the company offside with the regulators{'} requirements and could result in the company{'}s shares being placed on cease-trade. 


%------------------------------------------------------------------------------------------------
\subsection{Cryptoassets and firms that hold them}
The central component of a financial report is the firm{’}s balance sheet which lists assets and liabilities. We use the term \textit{cryptoasset} and \textit{cryptoliability} to refer to items whose value are contingent on blockchain technologies. Cryptoassets include cryptocurrencies and other tokens of tradeable value. Without loss of generality, we assume cryptocurrencies are native to their underlying blockchain, can be transacted directly, and are used to pay fees for transaction execution, \eg Bitcoin and Ether. For blockchains that allow developers to deploy custom decentralized applications, the applications might issue and manage the ownership of custom tokens. In the case of Ethereum, these tokens are often \textit{ERC20} tokens~\cite{erc20}, where \textit{ERC20} is sometimes misunderstood to mean what the token represents; rather, \textit{ERC20} is a technical standard about how token transactions are invoked.

What a token represents varies across applications but includes at least one of the following major categories. \textit{(1) Access tokens}: a service is developed which requires its own custom tokens for using the service; \textit{(2) Backed tokens}: a token issuer claims to be holding something valuable (material or digital) in reserve, and the token represents a claim on these reserves; \textit{(3) Equity tokens}: a firm issues tokens to represent ownership shares of the company; and \textit{(4) Collectable tokens}: the token itself is offered as a contemporary collector{’}s item (\eg \textit{ERC721}~\cite{erc721}). Digital tokens of these types predate blockchains; for example, \textit{Linden dollars}, \textit{E-Gold}, and \textit{(4)} URLs are arguably examples. Equities \textit{(3)} are almost entirely dematerialized (\eg paper stock certificates are rare) but operate closer to \textit{(2)} with a central depository.

Tokens and cryptocurrencies are issued to an initial set of owners through any mechanism of the issuer{'}s choosing. A popular option is auctioning a set of tokens to the public in an ``Initial Coin Offering'' or ICO. This is intended to resemble the ``Initial Public Offering'' (IPO) of a firm{'}s stock, but ICOs often lack any consumer protection or regulatory compliance. An ICO of an access token might raise capital for developing or improving the service that will use these tokens. Buyers obtain such tokens to lock in the purchase price (hedging) or for speculation. 

Any action that results in a firm borrowing a cryptoasset results in an offsetting cryptoliability. For example, an exchange service that holds bitcoin or tokens on behalf of its clients has both an asset (the cryptoasset it holds in its possession) and a liability (the obligation to repay its clients). Tokens could represent a debt instrument, like a bond, commercial paper, or repurchase agreement. In a recent pilot, a blockchain-based certificate of deposit was issued~\cite{NBCJPMorganBlockchain}. This creates an asset for the investors (Western Assets, Pfitzer, etc.) and a liability for the issuer (National Bank of Canada with J.P. Morgan). 

Bitcoin and Ether are native to their blockchains. Other assets or liabilities are digital representations of an off-blockchain asset or liability. One example would be the recognition of a land deed on a blockchain in the form of a token. The blockchain must be invoked during the audit to validate ownership, the land itself is obviously not on the blockchain.

To illustrate how a firm might come to possess a cryptoasset, consider four examples of firms that have actually sought audits or regulatory exemptions from audits\footnote{[Anonymized] Based on data provided to us by the financial regulator in the region where the authors work.}. A mining firm will invest in specialized computing equipment and electricity to generate cryptocurrencies it will hold as assets. An exchange service will hold demand deposits of cryptocurrencies or tokens, as well as governmental currencies, and allow its users to trade them. An investment fund will hold a portfolio of cryptoassets (\eg TIQ101-CF is 50\% bitcoin, 35\% ether, 15\% litecoin \todo{[cite-confidential\?]}) and sell shares to investors through a standard financial platform. Finally, a firm initiating a token sale will raise capital in cryptocurrencies, held as an asset, in return for tokens that are not generally liabilities (it may also reserve some tokens as an additional asset).

% [c] : https://3iq.ca/wp-content/uploads/2017/3iq-om-summary-en.pdf  <-- is this confidential? it is on public web but it says not to be distrubuted in public? 

%------------------------------------------------------------------------------------------------
\subsection{Presentation of Cryptoassets}
A financial report will classify assets and liabilities according to categories established by the auditing standards adhered to in the firm{'}s reporting country (American Generally Accepted Accounting Principles in the United States and International Financial Reporting Standards (IFRS) in many other countries). The following discussion will apply IFRS although the considerations under US GAAP are similar. \todo{ GAAP and IFRS citations  }


\paragraph{Cryptoassets.} IFRS does not provide specific standards for how to account for cryptoassets therefore their presentation is based on existing standards that were not conceived with the nature of these assets in mind. This results in a presentation of cryptoassets that does not necessarily reflect the underlying economic reality of the assets at hand. For instance, an investment portfolio that holds a combination of Bitcoin and Google stock for long-term capital appreciation purposes, would report the Bitcoin as an intangible asset~\cite{RCGTIFRsCrypto2018} and the Google stock as a short-term investment. The Google stock would be presented at its fair market value and any gains and losses on this investment would be immediately reported as part of net profit. 

The Bitcoin, on the other hand, would be presented as an intangible asset. This investment could either be presented at historical cost (what the company paid for the Bitcoin) or revalued to its fair market value. If the Bitcoin is reported at cost, any increases in value would not be recorded until the asset is sold (decreases in value below cost would be reported immediately). If the Bitcoin is reported at fair market value, any increases in value would be reported in Other Comprehensive Income, a special category that does not fall into net profit. This means that while the investment fund holds two investments, Bitcoin and Google stock, both for long-term growth purposes, the current standards would not allow these investments to recorded on the same basis. This also means that increases in value of Bitcoin or other cryptoassets are not recorded as part of net profit, which obfuscates a company{'}s annual reported earnings and undermines their usefulness for investors. 

\paragraph{ICOs.} Currently, there is no guidance on how to account for proceeds raised during an ICO. During an ICO, a company issues its tokens in exchange for fiat or other cryptoassets, depending on the offering. However, the presentation of these funds will depend on what the ICO holder is entitled to receive in exchange for whatever they have given up.  

If the tokens received represent  a residual interest in the issuing firm, the tokens would be presented as part of share capital, like would be the case for traditional share issuances. However, most ICOs are not set out to give the holder an interest in the company as a whole, but rather represent an interest in a specific project. If the holder is owed some type of obligation, like access to a marketplace or participation in an, then recognition of the funds received as a liability would be appropriate. 

Under rare cases, the proceeds could be reported as revenue if the funds received do not qualify as either share capital or represent some ongoing obligation towards the ICO holder. However, we expect these instances to be fairly rare as it is unlikely that an investor would contribute to an ICO and not expect something of value in return. 

\paragraph{Legal Status.} One challenge for regulators can be determining whether or not a token is, in fact, a security for legal purposes and, therefore, whether or not regulation applies. Existing securities law is not clear on how to classify tokens and different classifications will apply to coins with different characteristics. To date, not a single ICO has been approved by the SEC~\cite{SECICO2017}. In Canada, securities regularities have approved several ICOs through a regulatory fast-track program~\cite{fastTrackICOImpakt}. Despite a desire for Canadian regulators to foster innovation through fast-track programs, securities regulators remain skeptical and have issued a notice cautioning investors of the risks in this sector~\cite{CSAICO2017}. 

%------------------------------------------------------------------------------------------------

\subsection{Takeaway}
Several parties, including auditors, management and regulators, have an interest in ensuring that entities in the crypto space can obtain an audit, namely to satisfy regulatory requirements in order to attract capital. However, issues surrounding the presentation and measurement of these items on the financial statements undermine the potential information content of these statements. 

The following sections will present key issues that currently are troubling auditors and provide solutions or directions for future research. 

