% !TEX root = ../main.tex
\section{Key Issue: Existence of Assets and Occurence of Transactions}

Auditors need to establish that assets and liabilities reported at a point in time are real and that the transactions reported over the year did, in fact, take place. The auditor must also ensure that the transactions were neither fraudulent nor illegal and had a legitimate business purpose.  

%---------+++++++++++++++++++-----------------
\paragraph{Issue: Meta-information.} Assets and liabilities that are transacted on a publicly readable blockchain record basic details, like times and values, but auditors require further information to validate the nature of a transaction. For example, assume an employee{'}s salary is paid in Bitcoin. Bitcoin{'}s blockchain shows that a transaction occurred but it does not specify it as a salary; nor does it confirm basic details like the number of hours paid, if the nominal amount is BTC or a spot conversion from a governmental currency, and what deductions for tax or benefits were applied. Most importantly, it does not confirm the amount the employee was paid accurately reflects the number of hours worked at the authorized pay rate. Therefore, the company either requires a verbose secondary ledger to track these details or a Decentralized Application (DApp).

%---------+++++++++++++++++++-----------------
\paragraph{Issue: Off-Blockchain Transactions.} Transactions may involve the exchange of cryptoassets for off-blockchain assets. For instance, a company may pay its supplier in Bitcoin for a shipment of raw materials. Although the Bitcoin is blockchain-native and that side of the transaction benefits from the consensus algorithm, the blockchain cannot verify that the right quantity or quality of raw materials were received in exchange for the consideration paid. Therefore, given that half of this transaction occurred off-blockchain, it would need to be audited like a traditional raw materials purchase. 

%---------+++++++++++++++++++-----------------
\paragraph{Issue: Finality.} While blockchains are touted as immutable, the finality property (like all security properties) is subject to assumptions. Immutability of a blockchain is subject to consensus taken across miners according to computational ability. Consensus is not instant: a transaction might be included and then quickly dropped as consensus forms between different proposed chains. And it is never guaranteed to be final: an agreement within the computational majority of miners can unroll past transactions, For example, Ethereum{'}s miners branched the main blockchain to modify some past transactions to reverse a hacked DApp transactions~\cite{dupont2017experiments}. The challenge of finality was an important design consideration in the Bank of Canada{'}s Project Jasper~\cite{chapman2017project} which sought to establish a Distributed Wholesale Payment Systems. ``Project Jasper was structured so that a transfer of [digital currency] was equivalent to a full and irrevocable transfer of the underlying claim on central bank deposits.'' Given that on the Ethereum blockchain, transactions could be reversed, the work around for Project Jasper was to implement a ``design feature (which) relates to the issuance of (digital currency) and is therefore independent of the platforms upon which Jasper was built.''.

The issue of revocability is not uncommon for accountants. In many cases, sales agreements provide customers with the option to return merchandise within a pre-established period. When recording revenue, the accountant much estimate the percentage of returns and factor this into the amount of revenue to be recorded. The same concept could be applied for the issue of finality. The firm could estimate the amount of returns or reversals that are likely to occur and factor this into their transaction recognition. Alternatively, auditors could establish a generally accepted threshold (for instance, number of days) after which a transaction would be considered final.  %we could discuss the practice of transaction confirmation for 6+ blocks here as well

%---------+++++++++++++++++++-----------------
\paragraph{Issue: Completeness.} Firms generally do not hide assets as this undermines their reported financial health, however a firm might hide an asset to shift its acquisition forward in time. Additionally, hiding liabilities promotes a firm{'}s solvency. Therefore, auditors are charged with determining that they have obtained the full measure of a firm{'}s transactions to ensure the completeness of the information under analysis.

In a blockchain-enabled world, the blockchain contains the record of all transactions carried out during the year. If the auditor has a list of all the keys that belong to the entity under audit, they can easily obtain an account of all the transactions carried out on the blockchain during the period. However, this still raises the issue that the client may have entered into side-arrangements with related parties. While these challenges are magnified on a blockchain due to the pseudo-anonymity of this environment, the underlying challenge remains the same as it would in a traditional audit.

There is always the possibility that, for instance, a client has not reported all keys in his possession to the auditor. Therefore, transactions on those keys would not be part of the known set of information under audit. However, this situation is not unlike a traditional audit. The client may have unreported bank accounts at a different bank than their usual institution that the auditor would not know about. 

%---------+++++++++++++++++++-----------------
\paragraph{Issue: Transaction Complexity.} Bitcoin and Ethereum transact native currency according to established protocols, and Ethereum-based tokens gravitate toward standards as well. However nothing prevents DApps from transacting in complex ways. For example, one proposed DApp for crowdfunding projects would allow stakeholders to split off into a smaller DApp, taking a share of the assets from the parent DApp with it~\cite{dupont2017experiments}. Another example would be the Lightning Network~\cite{poon2016bitcoin} which is a second layer, off blockchain payment network that was implemented for scaling Bitcoin to increase its throughput. The Lightning Network is more complex than simple Bitcoin payments and involves payment channels and routings.
 
While firms may complicate their transactions either for economic benefits in order to obfuscate the true nature of their transactions, the onus is on the firm under audit to operate in an auditable fashion. In other words, while it is the auditor{'}s responsibility to design procedures to gain comfort over an entity{'}s operations, it is the responsibility of management to implement controls and procedures to ensure that the entity is audit-ready. Therefore, some complex operations, while economically sound, may be avoided to ensure the audibility of operations. 

%---------+++++++++++++++++++-----------------
\paragraph{Issue: Transaction Pointers.} Blockchain transactions need to be uniquely identified to be pointed at by secondary financial records. Due to an implementation fault in Bitcoin, transactions broadcasted with one identifier might end up in the blockchain with a different identifier -- Transaction Malleability --, which could lead a firm to conclude a transaction did not take place when in fact it did, but under an unexpected identifier~\cite{andrychowicz2015malleability}. Some firms went bankrupt when their automated system kept honouring refund requests from a malicious entity claiming the refunds were not going through~\cite{trautman2014virtual,decker2014bitcoin}. The larger lesson here is that auditors cannot always safely abstract away low level implementation details.   

%---------+++++++++++++++++++-----------------
\paragraph{Issue: Occurrence.} \todo{This contradicts some of the points mentioned in Transaction Pointers and Finality, also the point made here is also mentioned there. My suggestion is to remove this Issue or merge it with one of the above} In order to validate that cryptoasset transactions actually took place during the year, auditors will look to reproduce the transactions carried out. By obtaining a listing of the client{'}s cryptographic keys, the blockchain can provide a complete history of the transactions that occurred during the year. Due to the immutable nature of the blockchain, the auditor can be assured that the transaction did take place as described. However, the auditor will still need to validate meta-information surrounding the transaction (see above) to validate the transaction{'}s reasonability. 

%---------+++++++++++++++++++-----------------
\paragraph{Takeaway:} This section demonstrates that due to the immutable nature of the blockchain and its ability to report the totality of transactions conducted during the period, this technology provides a record upon which the auditor can obtain evidence to ascertain the occurrence of transactions. However, the auditor will continue to need to rely on external sources such as a verbose secondary ledger to validate the legitimacy and business purpose of those transactions. 


% = = = = = = = = = = = = = = = = = = = = = = = = = = = = = = = = = = = = = = = = = =

\section{Key Issue: Ownership}
In addition to being satisfied with the existence of cryptoassets, auditors must be satisfied that the assets reported on the company{'}s balance sheet do in fact belong to the company. For traditional assets, firms might store assets with a custodian. This does not eliminate the issue of ownership, it simply shifts the concern from the firm{'}s audit to the audits of central custodians. Banks and organizations who provide custodial services are required to have robust internal controls over the safeguarding of assets in their care, and provide audited report supporting the reliability of their controls.

%---------+++++++++++++++++++-----------------
\paragraph{Issue: Cryptographic keys.} For most cryptoassets, the asset is considered owned by Alice if Alice possesses a private signing key that can be used to sign a transfer of the asset. Through decentralized apps, alternative notions of ownership are possible to define, but this idea of a signing key is foundational and seen in bitcoin, ether, ERC20 tokens, etc. (not to mention earlier e-cash proposals dating back to the 1980s). Thus demonstrating knowledge of this key is necessary (but not sufficient, as we will discuss shortly) to demonstrating ownership. The most direct technique is to use a zero knowledge proof of knowledge of this private key, and staple in some information identifying the context of the proof. For standard proofs, this is cryptographically equivalent to simply signing a challenge message with the key\footnote{A Schnorr Sigma-protocol with the challenge hashed in using Fiat-Shamir is exactly a Schnorr signature and closely related to an ECDSA signature.}. Folklore protocols of sending small cash amounts from an allegedly owned account to the auditor to demonstrate control are also commonly noted in the literature (and used in at least one occasion \todo{[cite]}). This offers similar security but adds ethical complexities for the auditor.
\todo{ It requires more cryptography citations here!  }
% What to cite for the [cite] placeholder
We note that while this cryptographic proof is necessary, it is not sufficient. What it proves is that the purported owner has access to the person holding the signing key. A malicious company might arrange for the owner of cryptoassets to engage in signing statements or moving test amounts fraudulently on their behalf. This issue is not new: an insolvent retail store might borrow inventory from elsewhere to inflate its assets during an audit, which will in this case involve a physical visit and inventory check by the auditor. Auditors mitigate this by arranging a common date for all audits of physical inventory, and similarly, cryptographic audits might be synchronized on a fixed schedule to prevent the same assets from being counted for different companies in different audits~\cite{dagher2015provisions}. 

%---------+++++++++++++++++++-----------------
\paragraph{Issue: Design Transparency.} For cryptoassets that are native to a blockchain, like Bitcoin or Ether, ownership is implemented at the protocol level and has been vetted over the lifetime of the blockchain. However many assets are created and owned through decentralized apps. Coding standards, such as the ERC20 token standard in Ethereum, might be followed but this standard only specifies necessary ways of transacting the tokens, not the mechanics of what ownership means. These mechanics can become complicated. For example, the DAO maintained ether and two types of tokens (DAO tokens and reward tokens) across four different internal accounts, and holders of DAO tokens could split off balances from these accounts into a new (two token, four account) DAO, in addition to certain types of withdrawals. Establishing ownership requires understanding the internal accounting of the applications maintaining the assets. 

%---------+++++++++++++++++++-----------------
\paragraph{Issue: Self-custodianship.} Cryptocurrency advocates point to its non-reliance on trusted third parties as key to its appeal \todo{[cite: CHI paper]}. Thus, the use of a custodian for cryptoassets (or more specifically, the private keys controlling the assets) is controversial. The advantage of a custodian is that it might more quickly specialize in security than end users, and this is certainly true for some users \todo{[cite: MIT]}. Self-custodianship of non-digital assets, such as diamonds for a jewelry retailer or cash for a currency exchange, is already a concern for financial auditors. Further, custodianship over cryptographic keys is a financial factor in other sectors, such as certificate authorities like Versign or Symantec which maintain keys critical to HTTPS and DNSSEC \todo{[cite]}. To date, no blockchain custodian or exchange has been able to produce a report that supports the reliability of their internal controls in order to provide auditors with comfort over the sufficiency of their systems. Therefore, auditors cannot rely on the internal controls present at custodians to obtain comfort over the ownership assertion. 

%---------+++++++++++++++++++-----------------
\paragraph{Takeaway:} In order for auditors to validate ownership, they must rely on cryptographic proofs as a first step. However, in order to avoid double-counting of keys, an industry standard common date should be arranged to provide a generally agreed upon ``state of the world'' where keyholders can demonstrate ownership. 

% = = = = = = = = = = = = = = = = = = = = = = = = = = = = = = = = = = = = = = = = = =
\section{Key Issue: Valuation}
When values are reported on financial statements, they must be reported in the functional currency of the firm, meaning the primary governmental currency used. A challenge for blockchain entities is to determine the valuation of cryptoassets on the financial statement date or the conversion rate for sales and expenditures made throughout the year. Auditors must be satisfied that the values reported in the financial statements are accurate and represent the underlying economic reality. While fair valuation is important for establishing a firm{'}s financial health, it is also important for auditors to use in determining what to focus on during the audit itself. The objective of an audit is to certify that the financial statements are free of material misstatement. This does not mean that the statements are free of all errors; it merely means that the statements are free of errors that could substantially change the opinion of an informed user. 

%---------+++++++++++++++++++-----------------
\paragraph{Issue: Fair value.} A significant obstacle for obtaining audited financial statements is the determination of a fair value for cryptoassets. For foreign currencies, firms value them at the closing rate on the transaction date, as reported by the Central Bank in which the firm operates. No universal central bank offers rates for currency-like cryptoassets. At the time of writing, one Fortune 500 financial firm, CME, offers a daily reference rate for Bitcoin, but not other cryptocurrencies or cryptoassets. Without an authoritative reference rate, current market quotes may be considered. According to IFRS 13~\cite{ifrs201113}, fair value is measured using the ``principal market for the asset or liability; or  in the absence of a principal market, in the most advantageous market for the asset or liability.'' The principal market for a cryptoasset may not be apparent. For instance, a company may purchase cryptoassets from several different exchanges or parties. 

%---------+++++++++++++++++++-----------------
\paragraph{Issue: Bans.} There regulatory rules for cryptoassets differ based on the jurisdiction; for instance, cryptoassets are absolutely banned in Algeria, Bolivia, and Pakistan (among others), and implicitly banned under investor protection rules in Colombia, Saudi Arabia, and China (among others)~\cite{cryptoLegalityWiki}. Also there are banking limitations for companies dealing directly with cryptoassets in some other countries such as Canada, India and Thailand. This limitation impacts the valuation of cryptoassets on hand. For instance, assume a company acquires a token when it is legal to do so, and several months later, the government prohibits companies in that nation from holding that token. While this token may be traded in other parts of the world, the fair market value to its holder in the restrictive nation is arguably zero as their ability to use or sell the token is limited. Therefore, when determining the fair value of assets at hand, the auditor must consider the regulatory environment surrounding cryptoassets. 

%---------+++++++++++++++++++-----------------
\paragraph{Issue: Inadequate Liquidity.} While bitcoin and ether enjoy around-the-clock trading across many markets, lesser known coins, tokens, assets, or liabilities may trade slowly, in low volumes. Generally speaking, low liquidity results in stale last sale prices, and large bid-ask spreads. This challenging but not unprecedented in financial auditing: privately held stocks and over-the-counter financial instruments share a similar profile.

%---------+++++++++++++++++++-----------------
\paragraph{Issue: Geographical Variation.} Because bitcoin can be moved digitally without any geographic restrictions, it should be the case that any variation in bitcoin prices across exchanges would be consumed by arbitrageurs. However, empirically this is not the case due to fees, settlement delays, and other frictions with exchange services \todo{[cite]}.

%---------+++++++++++++++++++-----------------
\paragraph{Issue: Fungibility.} Blockchains generally preserve a full transaction record for each (division of a) cryptoasset. This may lead to price discrepancies between equal amounts of cryptoassets, where assets with clear provenance might be preferred to assets with long transaction records involving unknown entities and possible fraud, theft, or other issues that might legally encumber the current holder of the asset. The materiality of a premium for ``clean'' cryptoassets is an open research question; some markets for fresh bitcoin were announced but never materialized \todo{[c]}. 

%---------+++++++++++++++++++-----------------
\paragraph{Issue: Volatility.} Assuming the valuation of cryptoassets can be made, it is important to consider how stable this valuation will be over time. The cryptoasset with the richest historical data is bitcoin and by any standard measure of volatility, its volatility currently exceeds the most traded currencies and commodities. If a firm{'}s balance sheet consists of a large unbalanced\footnote{An exchange service that holds bitcoin on behalf of its users will have the bitcoin listed as both an asset and a liability, thus they are balanced and price movements have no impact on their solvency.} cryptoassets or cryptoliabilities, price movements significantly impact its financial health and even its solvency. A comparable scenario is a financial firm holding exotic, volatile securities or derivatives---in these cases, auditors might ``stress test'' the firm{'}s balance sheet under different valuations. Starting with bitcoin and moving to other cryptoassets, researchers should consider procedures for the ``shelf life'' of valuations and realistic stress tests.

%---------+++++++++++++++++++-----------------
\paragraph{Issue: Financial Projections.} Financial statements are prepared under the assumption that the reporting entity will continue operating for at least 12 months after the balance sheet date. Auditors are required to perform procedures to verify the reasonableness of this assumption. For instance, the auditors could examine projections where the cryptoassets the firm holds increase or decrease in value, however this may be difficult due to  the high volatility of cryptoassets. Bitcoin, for instance, increased 10-fold over the last half of 2017. The problem is even more substantial for companies that hold alt-coins which may have thin markets and short histories upon which to compare financial projections. A remedy would be the disclosure of significant assumptions underlying these projections to satisfy financial statement users of the company{'}s projected financial health. 

%---------+++++++++++++++++++-----------------
\paragraph{Takeaway:} While the valuation of cryptoassets with large trading volumes like Bitcoin and Ether would be fairly straightforward by referring to values in an active market, the valuation of coins with low liquidity may be more difficult. However, by using financial modelling using different assumptions for volatility, this challenge is not insurmountable.  

% = = = = = = = = = = = = = = = = = = = = = = = = = = = = = = = = = = = = = = = = = =
\section{Other Auditing Issues}

\paragraph{Issue: Reliability of Blockchain Records.} Another challenge for auditors is determining what type of evidence or records constitute sufficient and appropriate audit evidence. In other words, an auditor must be satisfied that the records on the blockchain are reliable. How can we satisfy auditors that blockchain records are reliable
 
\begin{enumerate}
\item[$\bullet$] How can we rely on blockchain? Has anyone tested it? For example, maybe Bitcoin is reliable due to large user base but what about more niche blockchain? How can I validate accuracy of mining procedures? Reliability of transaction records? How do I know that changes to blockchain code made and not undermine reliability?
\end{enumerate}

\todo{there are some comments on the google docs regarding this subsection that should be addressed(?)}

Traditionally, auditors depend on obtaining audit evidence from external parties as it is deemed as more objective or trustworthy than internal firm documents. For example, auditors will obtain confirmation from a bank of the amount of cash held at year end (a third party) rather than rely on internal records of banking transactions. Auditors traditionally rely on banks, investment companies, and lawyers as conventional sources of external confirmations.  However, a risk always exists that there is fraud or error at these institutions. 

%---------+++++++++++++++++++-----------------
\paragraph{Issue: Technical Literacy of Accounting Professionals.} To carry out an audit, auditors must possess sufficient knowledge in a subject area to understand the subject matter under audit and be able to question its underlying assumptions. This can be achieved through industry specialization, training or by relying on subject matter experts to provide knowledge in a particular area. Presently, many auditors are refusing mandates in the blockchain sector due to a lack of technological know-how regarding how to effectively carry out these audits. The risk of taking on such an audit (referred to as \textit{audit risk}) is too high, especially given numerous recent frauds in this sector. For instance, PinCoin raised over \$660M in an ICO before the management team vanished~\cite{techcrunchscammers}. However, this issue can be overcome by engaging multidisciplinary teams that can leverage the business knowledge of auditors with the technical know-how of computer scientists. These types of teams will also provide auditors with increased awareness of security threats.

%---------+++++++++++++++++++-----------------
%\paragraph{Issue: What is an expert?} Oftentimes auditors rely on experts to provide assistance in auditing highly specialized asset classes. For instance, auditors often engage Chartered Business Valuators to assist in the evaluation of property like land and buildings. These professionals are engaged due to their subject matter expertise and experience in the field. In the blockchain sector, it is unlikely that any one individual would constitute an expert for audit purposes. Given that this sector has barely been in existence for a decade, if may be difficult to assess what constitutes an experienced or non-experienced professional. Furthermore, auditors are required to perform procedures to assess the assumptions used by the expert. The more likely alternative would be to engage multidisciplinary teams that have expertise in cryptography, security and business to assist the auditor in obtaining comfort over specialized blockchain assets and liabilities. 

%---------+++++++++++++++++++-----------------
\paragraph{Issue: Financial Literacy of Technology Professionals} While auditors must learn more about the technology, managers of companies in the blockchain space must also learn about the internal controls required to safeguard against errors, misappropriation of assets and fraud. Given that financial statements are ultimately the responsibility of management and that senior management must certify the internal controls as part of the financial statement disclosures, managers have an incentive to implement a control environment that promotes a strong tone at the top, ethical conduct, and oversight of the financial reporting function.

%---------+++++++++++++++++++-----------------
\paragraph{Issue: Internal Controls.} It is necessary for a firm to be able to demonstrate the existence and ownership of its crypto-assets, however it must further demonstrate that it has adequate procedures in place to prevent or detect fraud and theft. Consider a firm holding DAO tokens. Internally, it must ensure proper controls over the signing keys these DAO tokens are assigned to. External to the firm, the DAO application maintaining these tokens on Ethereum{'}s blockchain must itself be secure (which in the case of the DAO, it was not). Auditing internal procedures over cryptographic keys is not unprecedented---maintaining certificate authority keys is critical to some security firm{'}s financial prospects (a security breach at one firm, \textit{DigiNotar}, bankrupted it~\cite{zetter2011diginotar}). Symantec{'}s self-custodianship over these keys involves elaborate security ceremonies whenever they are required \todo{[Cite]}. 

%---------+++++++++++++++++++-----------------
\paragraph{Issue: Materiality.} Materiality is considered to be any value that is significant enough to have an impact on the user{'}s decision-making, and GAAS dictates that materiality for profitable entities be set at 5\% of pre-tax income. This means that, in aggregate, the sum of identified and extrapolated errors identified by auditors during their audit cannot exceed 5\% of pre-tax income for a clean opinion to be issued. However, companies may manipulate their selection of accounting policies to influence their net income in order to increase materiality. This means auditors would only look at items over a higher threshold, leaving smaller amounts subject to a lower level of scrutiny and greater opportunity for fraud or error. However, this issue is possible for all clients and therefore auditors would be aware of and design procedures to address this risk for all mandates.
 
%---------+++++++++++++++++++-----------------
\paragraph{Issue: Forks and Airdrops.} Existing accounting standards do not contemplate how to account for non-reciprocal transfers of assets like in the case of forks or airdrops. For example, in August 2017, when a hard fork created Bitcoin Cash from Bitcoin, holders of Bitcoin had two types of assets on hand. Bitcoin Cash was not paid for but resulted from a split between the two currencies. However, if existing accounting standards require the measurement of transactions at historical cost (what was paid for the assets), then recipients of Bitcoin Cash would report this new asset on their books at a value of \$0. Certainly, this does not represent the true value acquired through the fork. Therefore, auditors must address the issue of an accounting standard that does not contemplate how to measure the value of cryptoassets when considering whether the financial statements they are reporting on are accurate in all substantial respects. More commonly for ERC-20 tokens on Ethereum, the recipient of the tokens don{'}t have the option to reject the deposits nor would get notified of the new tokens received, which could result in issues with completeness. 

%---------+++++++++++++++++++-----------------
\paragraph{Issue: Detection of Fraud.} An auditor conducting an audit in accordance with GAAS is responsible for obtaining reasonable assurance that the financial statements taken as a whole are free from material misstatement, whether caused by fraud or error. Owing to the inherent limitations of an audit, there is an unavoidable risk that some material misstatements or fraud may not be detected, even though the audit is properly planned and performed in accordance with GAAS. Due to the complex and rapidly changing nature of this sector, auditors are especially weary of fraud risk in this area. 
However, not all frauds that occur this space are new kinds of fraud. For instance, Ponzi schemes which have existed since they were perfected by Charles Ponzi in the early twentieth century~\cite{ponzi2001rise}. In 2017, OneCoin was found to have raised over \$350M of funds from investors for an ICO that was a Ponzi scheme~\cite{atlanticCryptoPonzi2017}. Stories like this only reinforce the perception in the auditing community of the dangers of operating in this space. However, auditors can protect themselves by ensuring that they only work with reputable clients. Managers can demonstrate their commitment to sound business practices by adopting controls that ensure KYC and AML rules are adhered to.

% = = = = = = = = = = = = = = = = = = = = = = = = = = = = = = = = = = = = = = = = = =
\section{Discussion and Research Agenda}
\todo{Check out the notes on the google docs regarding this section}

Overall, this paper has demonstrated that, in comparison to traditional audits, audits of clients that hold material amounts of cryptoassets are complex but not impossible. For instance, while current accounting standards result in the presentation of an accounting fiction that is not always linked to the underlying use of crypto, adequate note disclosure can provide financial statement users with sufficient information in order to understand the underlying crypto operations. 

From our discussions with practitioners, existence, valuation and ownership were often cited as the three main stumbling blocks to providing an audit opinion. However, we argue that these issues are not insurmountable if industry guidelines are put into place to allow auditors to verify their client’s cryptographic keys against a “state of the world” at a generally accepted point in time. Verifying existence and ownership largely hinges on an auditor’s ability to verify the contents of a client’s cryptographic keys. However, the auditor must be certain that these keys in fact belong to the client and do not simply represent access to an account. Once ownership has been proven, the auditor can rely on the immutable properties of the blockchain to verify existence as the blockchain provides the entire record of transactions since the blockchain’s inception. The issue of key sharing is important but is not unlike a situation in the real world where a related party could give the entity under audit a large sum of cash to hold at year end and report on their financial statement to buoy their financial performance. 

Volatility does complicate the valuation of alt-coins and other coins with low trading volumes. However, many other exotic securities exist where accountants rely on complex financial modelling to determine a price. 

In sum, this paper argues that although auditors are right to be cautious to enter a new sector where clients not be initiated to the importance of internal controls and where numerous frauds have recently been perpetrated, audits are possible. Audit risk can be reduced through proper vetting of clients and management teams. Alternatively, accounting firms can provide advisory services to clients in anticipation of going public in order to ensure that these clients implement robust internal controls to support their financial reporting function. 

% = = = = = = = = = = = = = = = = = = = = = = = = = = = = = = = = = = = = = = = = = =
\section{Concluding Remarks}

\todo{Check out the notes on the google docs regarding this section}

% Non-trivial
% Not infeasible
% Onus on firm not auditor -> move away over time

