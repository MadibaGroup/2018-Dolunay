% !TEX root = ../main.tex

\section{Introduction}
% Why is this important.
% PAPER BEING SOLD HERE


%______________________________________________________________%

\section{Preliminaries}

\subsection{What is Front running?} %Definition of frontrunning
\label{sec:What is frontrunning?}

\emph{Front running} is analogous to any course of action during which a person benefits from prior access to inside or confidential market information to upcoming transactions and trades. This problem can occur in both financial and non-financial systems, however, it is more noticeable within the trading and exchange systems as it was originated back in stock markets. In the old times, all the trades were executed on papers, so in case of receiving a large order from a client, the broker might say this loud and other people around the table could be informed. Therefore, a malicious trader would now run in front of that order and put his own transaction in between (before the trade was executed) and profit from the price increase of the stock. In other words, a group of market participants obtain non-public market information which allows them to front-run other users trades by putting their orders a head of those trades and benefit from advance knowledge of pending client orders. The two significant factors which cause the front-running practice to happen within the financial markets are (i) imperfect competition and (ii) liquidity uncertainty. \textbf{should I explain more?!} Any sort of front-running activity within the financial markets is considered as an unethical and illegal practice as it is unfairly beneficial for a few market participants who have the privilege of acting on this information and taking advantages at the expense of the investor. 



\subsection{Front Running on the Blockchains}

Blockchain technology has received an exceptional amount of attention since it was first introduced as the underlying technology of the cryptocurrency Bitcoin in Satoshi Nakamoto's (pseudonymous) 2008 white paper \cite{nakamoto2008bitcoin}. Many decentralized applications that are nowadays built on top of this technology represent the significance of the blockchain as it completely eliminates the single points of failures within any systems. However, blockchains have some inherent characteristic that leave them vulnerable to \emph{front running} behaviour. Although this data structure is known to be publicly visible to every network participants, information is layered  and some of them are accessed only by insiders. Any network participants that runs a full node on the blockchain is able to obtain those information and as mentioned in Section~\ref{sec:What is frontrunning?}, this leaves the application vulnerable to front running. Decentralized exchanges are a group of blockchain financial applications where front running can be executed. Bancor \cite{hertzog2017bancor} is an example of such systems in which front runners benefit from potential price increase of the market stocks (details are provided in Section ...). Decentralized namespaces, as blockchain non-financial applications, are another example. Ghazal \cite{moosavighazal} is such system in which upon seeing a transaction from Alice to register \texttt{alice.com}, front runner can go ahead of her and register this domain name. He can later sell \texttt{alice.com} to Alice for a higher price and benefit from the price increase.
%___________________________________________________________%

\section{Related Work}
As the traditional frontrunning was originated to trading financial instruments most of the literature are focused on financial aspects of the markets. %TODO: expand this with something


\subsection{Historical Context / Classic Frontrunning}
Frontrunning was first appeared on the Chicago Board Options Exchange ( \textit{CBoE}) ~\cite{markham1988front}, and was identified by \textit{Securities Exchange Commission} in 1977 as the following:
\begin{quote}
The practice of effecting an options transaction based upon non-public information regarding an impending block transaction\footnote{Block in the stock market is large number of shares, 10\,000 or more, to sell which will heavily changes the price} in the underlying stock, in order to obtain a profit when the options market adjusts to the price at which the block trades. ~\cite{sec1978optionsmarket}
\end{quote} 

%In actuality, front-running is more complex than this definition suggests. It encompasses at least three forms of conduct, each of which raises different regulatory and policy issues.9 They are: (1) trading by third parties who are tipped on an impending block trade ("tippee" trading); (2) transactions in which the owner or purchaser of the block trade itself engages in the offset- ting futures or options transaction as a means of "hedging" against price fluctuations caused by the block transaction ("self-front-running"); 1 ° and (3) transactions where a broker with knowledge of an impending customer block order trades ahead of that order for the broker's own profit ("trading ahead").
% from JerryWMarkhamFrontRunning.pdf

% More to read and add from : SECURITIES AND EXCHANGE COMMISSION Release No. 34-67079. pdf included in Related_Documents
% https://www.sec.gov/about/annual_report/1988.pdf

On the years after there were ongoing discussions between self-regulatory exchanges (\eg \textit{CBoE}) and  \textit{SEC} to regulate, detect and define laws and regulations to deal with frontrunning~\cite{markham1988front}, with \textit{SEC} stating: 
\begin{quote}
It seems evident that such behaviour on the part of persons with knowledge of imminent transactions which will likely affect the price of the derivative security constitutes an unfair use of such knowledge. \footnote{Securities Exchange Act Release No. 14156, November 19, 1977, (Letter from George A. Fitzsimmons, Secretary, Securities and Exchange Commission to Joseph W. Sullivan, President  CBoE).}
\end{quote} 

As defining what exactly is considered illegal front-running required more knowledge of how these new markets behave, \textit{CBoE} and other exchanges (and brokers) issued educational circulars for their members asserting that frontrunning violates existing rules, with some examples of what is considered frontrunning. However difference of opinion regarding the unfairness o front running activities, insufficient exchange rules and lack of a precise definition in this area resulted in no action by self-regulator organizations~cite{sec1978optionsmarket}. 

Further reading on the early details on the history and challenges of detecting and regulating frontrunning can be found in~\cite{markham1988front} . % TODO: add a more recent publication here as well! 

Initially the frontrunning policies only applied to certain option markets, later on in 2002, the rule was refined to include the same prohibitions to security futures~\cite{finra_2002}, which then in 2012 with the new amendment, FINRA Rule 5270, the frontrunning rule was extended to cover trading in an option, derivative, or other financial instruments overlying a security with some exceptions~\cite{finra_2012, sec2012frontrunning}. 


\subsection{Recent Literature on Blockchain frontrunning}

~\cite{malinova2017market}
~\cite{aune2017footprints}
~\cite{breidenbach2018enter}


% NOTE: use double-spend paper (first)
% talk about double-spend one kind of frontrunning (relationship between frontrunning vs double spend) rushing actors. 
% Double-spending fast payments in bitcoin.pdf


%______________________________________________________________%


\section{(On) Blockchain Front-running}

\subsection{Who Can Front-run?}

In general, all the network participants have the ability to front run specific transactions that are sent to the network. However, miners can include any transactions they like into the block they attempt to mine. Thus, they possess special power in terms of front running as opposed to other users of the network. In the following, we discuss and compare the two groups of blockchain potential front runners.

\subsubsection{Miners}
As mentioned above, since the Blockchain miners are the only parties who can decide on the order of transactions within a block they mine, they can easily intercept and reorder the pending transactions sitting in the mempool and profit from a guaranteed price-rise. For example in an Ethereum based application, a miner learns about the pending \textit{buy} transaction of 1000 Ether, presumably if this transaction goes through, it causes the price of Ether to increase. So the dishonest miner can step in front of this transaction and  place his own buy order ahead of it. He would simply create his \textit{buy} of 1000 Ether and include it within a block and now he mines the previous \textit{buy} transaction of 1000 Ether. Doing so, he would receive a better rate than other network users, can sell the assets he has received and gain a price advantage at the expense of others. Similarly, a dishonest miner can sell his tokens in front, if he sees a pending \textit{sell} transaction. In an open source blockchain application, there is no rules on how transactions must be ordered and miners are free to send transactions in the order they prefer. However, miners can only front run other transactions (by reordering them) within the block they happen to mine while they do not broadcast any transactions to the network. This makes the miners to be less noticeable to the network when front running.

\subsubsection{Blockchain users/nodes}
Any regular (non-miner or miner) user can also front-run other transactions in the network. For regular users to front run others on the blockchain, they need to be fully/well connected to other nodes on the network. Doing so, they are able to listen to the network and monitor all transactions that are broadcast. On the Ethereum blockchain users have to pay for the computations in small amount of Ether called \textit{gas}~\cite{AccountT67:online}. The price that users pay for transactions (a.k.a. transaction fees) can increase or decrease how quickly they are executed and included within the blocks by the miners.  This is because the Ethereum miners consume resources to process the transactions and so receive the transaction fees after creating the blocks. Once seeing two identical transactions with different transaction fees, profit maximizers miners are free to mine select the transaction which offers the highest fee. Therefore, any regular users who run a full-node Ethereum client can modify the order of pending transactions by paying a more amount of gas \ie by monitoring the network, upon seeing a  pending \textit{buy} transaction which will further increase the price of the asset, a font-runner user can pay a higher gas price and send his transaction a head of that. By doing so, he achieves a better rate from any other network users. Note that in this case, blockchain front runners are more visible to the network as they broadcast their transactions to all the network participants. 


%\subsubsection{Blockchain users/nodes}:
%\begin{itemize}
%\item{\textbf{Higher GasPrice}: Monitor transactions, rebroadcast with higher GasPrice}
%\item{\textbf{Fully/well connected nodes}: Similar to HFT situation of faster connections to other nodes, for a higher chance of transaction inclusion, also Sybil attacks?} 
%\item{\textbf{Noticable by the network}: More visible to the network as both transactions are broadcast to all the nodes}
%\end{itemize}
%
%
%\subsubsection{Miners}
%
%
%\begin{itemize}
%\item{\textbf{Their own mined blocks}: The can only frontrun transactions within the block they mine. This could be done by reordering the transactions within the blocks.}
%%^ If a miner or user sees an unconfirmed transaction in the mempool they could squeeze their own transaction to come in front and profit from that. Essentially, profitting from reordering the transactions.
%
%\item{\textbf{Less Noticable by the network}: Miners can include transactions within the blocks they happen to mine without broadcasting the transactions, hence less chance of being detected by the network participants.} 
%%Should the following be included? 
%%\item{\textbf{Incentivized mitigation methods}: applications could pay the fees to the miners to incentives them to behave } 
%\end{itemize}


\subsection{Historical evidence}
As blockchain records are immutably recorded, there is enough historical data to analyze for possible front running detection. For examples here we research some of the events of such attacks happening in the Ethereum blockchain applications.


% TODO: @shayan: talk about cancel order frontrunning on decentral exchanges (etherdelta), griefing 
% maybe inlcude 0xbitcoin ERC721 mineable tokens frontrunning!
% mint: https://etherscan.io/tx/0x9883bc3ad018fd2b649982f88fbad7dc5abcb8f11c4f1d87ef814ba30c2b3428


\subsubsection{Status ICO} %Frontrunning by a miner/mining pool

%@Shayan: Clean up this part, finish it! 

% ICOs and ceiling
% hard cap --> frontrun and buy the tokens in big chunk
% dynamic ceiling, fair token distribution --> status ICO


ICO, Initial Coin Offering, is one of the blockchain applications, specially blockchains such as Ethereum with smart contract capability. The common practice is to deploy a smart contract on the blockchain indicating the details of the ICO such as the trade ratio, when it starts and ends, and more details on how it will be capped.
In June 2017, Status.im started their ICO and within 3 hours they reached the dynamic ceiling in place that triggered the end of the ICO, summing in 300,000 ether in funds, estimated at more than 200 million dollars at the time of their ICO. ~\cite{statusicoanalysis}. The idea behind Dynamic Ceilings is to make it more costly for larger contributors,  in the form of transaction fees, which have to split their contributions to different addresses, with minimal impact to smaller contributors~\cite{statuswhitepaper}.
On the time of the ICO there were reports of Ethereum network being unusable and transactions were not confirming. Further study showed that some mining pools (\todo: define mining pool) could have been manipulating the network for their own profit.

%@shayan: make charts and figure out the orange seller

%NOTE: ICO Initial Coin Offering should be explained probably in introduction section. the process of ICO smart contract and how the capped system works --> limited time/funds can be included in the ico and why people would try to get in as soon as possible, explain how their investment strategy works and such. Maybe the second half could be explained here to say why frontrunning an ICO makes sense?
% add more on capped ICOs, status used Distribution & Dynamic Ceilings : https://blog.status.im/distribution-dynamic-ceilings-e2f427f5cca  ~\cite{statuswhitepaper}

% The dynamic ceiling and prevention of bigger investments getting their full investment in could be the reason behind the design of F2pool move, by having multiple addresses each contributing around 100ETH to the ICO smart contract, instead of having one big transaction in.
%Also the ICO smart contract only accepted transactions with gas price lower than 50gwei, unless they were whitelisted before. this was for preventing high gas payers to get in front of the line. although as this was not clearly communicated to users (UX issues?), there were tons of transactions that failed due to high gas price but also clogged the ethereum network as the miners chose transactions with higher gas price to be included in their blocks.


% weird findings: apparently all the tokens bought by f2pool was transfered to Yunbi, the chinese exchange: 0xB6c3647F55085B9a327404Fe8B718076Ee19245a

% Proof of f2pool involvement: https://etherscan.io/address/0x5fb8373bf5086fb38ef0fb686b3092145de0d1c1
% funds come from f2pool and refunds goes back to f2pool again

% I would like to spend some time to make charts from the transactions happening on the date of the Status ICO
% this will include: The blocks mined by F2Pool, containing transactions (~100 ETH) to Status ICO, other blocks mined in that time period, the movement of status tokens from those addresses allegedly linked to F2Pool (or/and Yunbi in this case).
% https://github.com/corpetty/ICO_analysis/blob/master/status/Status.ipynb
% https://github.com/corpetty/py-etherscan-api


%What is Bancor and how it can be frontrun:%

\subsubsection{Bancor} %frontrunning by network participants.
Bancor is an Ethereum-based application that allows users to exchange their tokens without any counterparty risk. This protocol aims to solve the cryptocurrency liquidity issue by introducing \textit{Smart Tokens}~\cite{hertzog2017bancor}-- ERC20 compatible tokens with a built-in liquidity mechanism that are always available to users. Smart Tokens can be bought and sold through the users smart contract at an automatically calculated price which displays supply and demand. Doing so, Bancor protocol provides continuous liquidity for digital assets without relying on an orderbook as there is no requirement to match sellers and buyers.

\par\noindent\textbf{Front-running Bancor} Recently, researchers have shown that Bancor is vulnerable to frontrunning attacks. Implemented on the Ethereum blockchain, when Bancor transactions are broadcast to the network, they sit in a pending transaction pool known as \textit{mempool} waiting for the miners to mine them. Since Bancor handles all the trades and exchanges on the Ethereum blockchain (unlike other existing decentralized exchanges), these transactions are all visible to the public for 16s (the average Ethereum blocktime) before being included within a block. This leaves this blockchain-based decentralized exchange vulnerable to the blockchain race condition attack as attackers are given enough time to front-run other transactions and gain favourable profits~\cite{BancorIs7:online}. Bancor frontrunning attacks can occur in two different ways:



%\begin{itemize}
%\item {\textbf{Miner Frontrunning.}} As mentioned, Bancor protocol uses an algorithm that automatically calculates the price of digital assets to provide better market liquidity. In the Bancor model, essentially buy orders increase the price of the tokens while sell order decrease it. Since the Blockchain miners are the only parties who can decide on the order of transactions within a block, they can easily intercept and reorder the pending transactions sitting in the mempool and profit from a guaranteed price-rise. For example, a miner learns about the pending \textit{buy} transaction of 1000 Ether, based on the Bancor design, if this transaction goes through, it causes the price of Ether to increase. So the dishonest miner can step in front of this transaction and  place his own buy order ahead of it. So he would simply create his \textit{buy} of 1000 Ether and include it within a block and now he mines the previous \textit{buy} transaction of 1000 Ether. Doing so, he would receive a better rate than other Bancor user, can sell the tokens he has received and gain a price advantage at the expense of others. Similarly, a dishonest miner can sell his tokens in front, if he sees a pending \textit{sell} transaction.
%
%
%\item {\textbf{Non-miner Frontrunning.}} Researchers have also shown that a regular non-miner user can also front-run Bancor. In order for the Bancor transactions to be executed on the Ethereum Virtual Machine (EVM),  users have to pay for the computations in small amount of Ether called \textit{gas}~\cite{AccountT67:online}. The price that users pay for transactions (a.k.a. transaction fees) can increase or decrease how quickly they are executed and included within the blocks by the miners.  This is because the Ethereum miners consume resources to process the transactions and so receive the transaction fees after creating the blocks. Once seeing two identical transactions with different transaction fees, profit maximizers miners are free to mine select the transaction which offers the highest fee. Therefore, any regular non-miner users who run a full-node Ethereum client can modify the order of pending transactions by paying a more amount of gas \ie by monitoring the network, upon seeing a  pending \textit{buy} transaction which will further increase the price of the asset, a font-runner user can pay a higher gas price and send his transaction a head of that. By doing so, he achieves a better rate from any other Bancor users.
%
%\end{itemize}



\subsubsection{Domain Name Registration} 

%"domain name frontrunning." The practice resulted in Network Solutions registering a previously-unregistered domain to itself immediately after someone searched for it, then holding the domain for four days before it could be purchased by someone else or at another registrar. But the company claims that it's merely trying to protect customers from others doing exactly that. Until there is more regulation over frontrunning from ICANN, this is the best it can come up with


%Following section talks about frontrunning in namespaces but limited to Ghazal. Will talk about NameCoin in the section of mitigations (commit & reveal method)
%
% PoC of frontrunning ghazal
Although frontrunning attacks have been more showcased in the context of decentralized exchanges and trading systems, they are are not yet limited to the financial markets. Frontrunning can occur within other blockchain based applications such as naming systems. Blockchain-based namespaces have been introduced to eliminate the role of central parties \ie domain name system (DNS) which introduces single point of failures in the entire web. One such system is Ghazal, an Ethereum-based naming and PKI system~\cite{moosavighazal}. Ghazal users rely on the Ethereum blockchain to register their \texttt{.ghazal} domain names and bind certificates to those names. In Ghazal model, a user would register domain name by executing the \textit{registerdomain} function from the Ghazal smart contract with the domain name in plain text as the function argument. As mentioned before, These transactions will sit in the mempool so that it would be mined by Ethereum miners and included in the block. During this period in which the transaction is not yet confirmed, frontrunning attack can occur by (i) dishonest miners and/or (ii) regular non-miner user. In the first case, a miner would intercept the \textit{registerdomain} transaction and register that name ahead of the user. A regular non-miner node in the Ethereum network can frontrun other user's \textit{registerdomain} at a good profit by paying higher transaction fees. In both cases the adversarial party could sell the domain name to the users for higher price.




%______________________________________________________________%




\section{How to stop frontrunning?}

\emph{Traditional Frontrunning Prevention Methods}. regulations/enforcement/broker education/sealed order
%@shayan: ^


\emph{Blockchain Frontrunning mitigation/prevention}. The traditional methods of preventing frontrunning are based on regulation and restrictions applied to the brokers and actors within the markets, such methods do not apply to blockchains, no enforcement methods, ... 

There are two main approaches to prevent front running, one to design a blockchain that is frontrun-resistent , and the other to design the application logics in a way that front running is not possible. 


\subsection{Frontrun-resistent Blockchain}
What does this mean to have a frontrun-resistent blockchain?  There are technical difficulties to achieve such solutions as there are unknown factors within such network designs (corner/edge cases).

\begin{itemize}
\item{\textbf{Privacy Preserving Blockchians}: Shielded transactions in ZCash do not reveal the sender, receiver, the amount and the data included in the transactions, hence they cannot be seen by network participants to be frontrunned. Although this limites the functionality of the blockchain/smartcontracts/dapps }
\item{textbf{Loopring/Dual Athoring}: Talk about this possible solution}

\end{itemize}


\subsection{Application design to prevent frontrunning}
Applications, in this case Dapps, could be designed in a way to prevent frontrunning.

\begin{itemize}
\item{\textbf{Commit/Reveal}: describe. General solution to prevent data leaks of the transactions. although it does not hide the participatory factor (it shows the actor participated in the application but hides the details, obvious participation(commit))}
% discussion about Anonymity-set ?
\item{\textbf{Submarine sends}: describe. General solution similar to commit and reveal. This is to solve the participatory factor of the commit and reveal solution.}

\item{\textbf{Application logic specific solution}: Depending on the use case of  the Dapp or the application, it could be designed in a way to deincentives some actors to frontrun the transactions or prevent them from doing so. As an example on desiging a decentral orderbook, it could be said that a miner could front run orders for financial gain on price improvement, however if the fees of the orders are sent to the miner of the block, it deincentiveses the miner to front run the orders as they already gain enough financial benefit from including the correct order of transactions. 
Another example for decentral orderbook design could be using call market design instead of  time-sensitive orderbooks. In such design the arrival time of the order does not matter as they are executed in batches. }

%Proof of burn methods ? 

\end{itemize}



%______________________________________________________________%




\section{Concluding Remarks}


% Could talk about a tool / a framework to detect frontrunning


%\subsubsection*{Acknowledgements.}
