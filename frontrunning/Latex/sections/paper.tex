% !TEX root = ../main.tex

\section{Introduction}



The concept of the \emph{front-running} goes back to the decades ago when financial instrument markets first started. Initially the trades required to be physically available in the exchange office to deliver the bids and offers, and traders, in the knowledge of future market change, would run in front of the others to get their orders in before the change in the market to profit from the change. Clearly in such scenarios front-running was possible yet not detectable as a damaging activity. However with the rapid advances of the technology in the last few decades, trading is not happening in person anymore and a greater number of applications are now vulnerable to front-running. Furthermore, Blockchain technology has introduced transparency and open markets which led to a new threat of front-running and it inherently affects all decentral applications, whether financial or non-financial. Arbitrage and Censorship, are also issues that had been linked to front-running attacks on blockchain. However, the goal of this paper is to provide a complete look at the front-running issues and thus the aforementioned will remain out of the scope of this research study. There has been little quantitative analysis and research of the front-running issue in decentralized applications, we aim to take a thorough look into the history of the front-running, discuss multiple existing issues in the eco-system and systematize the proposed solutions, and eventually match these solutions to their suitable use cases. \footnote{A note to the reviewer: this paper falls under "Case Studies (\eg of adoption, attacks, forks, scams \etc)" and also mentions methods for "Fraud Detection and Financial Crime Prevention". }

%For the FC: \textbf{CASE STUDY} should be used in here, also in abstract, also at the end of intro section.
% Why is this important.
% PAPER BEING SOLD HERE
%\textbf{Note to Change the structure of the paper: to pitch this paper as a case study, we can polish it as a Bancor case study: 1) here is the bancor project that we look at 2) we find that the issue is mostly the frontrunning 3) So we take a deep look into what is frontrunning 4) we find other case studies that are vulnerable and go through each case}


%Decentral applications are on the rise with Turing complete blockchains maturing to handle high monetary value transactions. Tokenization mushroomed to a whole ecosystem of tokenized applications on the blockchain, these applications inherently become a market as tokens are in the core of them. We started looking at some of these applications, in specific \textit{Bancor}~\cite{hertzog2017bancor}, to look into possible issues rising from these new models. Transparency of blockchains, although being a good feature of them, opens up these markets to some attack vectors that might not be visible on the time of the application design, one main attack is known as \textbf{front-running}, whether financial or non-financial. Arbitrage and Censorship, are also issues arises from the designs on the blockchain technology and had been linked to front-running attacks, hence we use the term \textit{front-running] losely to talk about any transaction manupulation with the intention of getting other transactions in before the original transaction takes place. In this paper, our focus is to have a complete overview of front-running on the blockchain while analysing Bancor issues. While looking at the possibility of front-running on other applications, we found interesting case studies that affects most of the top 25 decentral applications on Ethereum blockchain.



%______________________________________________________________%
%TODO: Censorship vs frontrunning ! can withholding a transaction be considered as frontrunning? 
%using frontrunning losely, we do not want to defining what exactly front running is, but talking about possibiities of attacks in this nature and also how to prevent these from happening .


\section{Preliminaries}

\subsection{What is Front-running?} %Definition of front-running
\label{sec:What is front-running?}

\emph{Front-running} is analogous to any course of action during which a person benefits from prior access to inside or confidential market information about upcoming transactions and trades.  The idea of front-running was originated back in stock markets. In the old times, all the trades were executed on papers, so in case of receiving a large order from a client, the broker might say this loud and other people around the table could be informed. Therefore, a malicious trader would now run in front of that order and put his own transaction in-between (before the trade was executed) and profit from the price increase of the stock. This problem occurs when a group of market participants obtain non-public market information which allows them to front-run other users trades by putting their orders a head of those trades and benefit from advance knowledge of pending client orders
Nowadays, there are a wide range of decentral applications implemented using blockchain technology. Being decentralized, there is no central version of information to be attacked, however, due to the inherent nature of the blockchain infrastructure, the data flow is transparent hence these applications are open to \emph{front-running} attacks. One such application is \textbf{Bancor} -- an Ethereum-based application that allows users to exchange their tokens without any counter-party risk. Researchers have shown that this financial blockchain application is open to a simple front-running attack.  However, front-running vulnerability is not limited to the financial application and it can happen in other types of blockchain applications as well. 


%The two significant factors which cause the front-running practice to happen within the financial markets are (i) imperfect competition and (ii) liquidity uncertainty~\cite{liang2005distressed}. Any sort of front-running activity wis considered as an unethical and illegal practice as it is unfairly beneficial for a few system participants who have the privilege of acting on this information and taking advantages at the expense of the investors. 

\subsection{How are Blockchain Applications Vulnerable to Front-running?} %which blockchain applications?
\label{sec:Front Running on the Blockchains}

%Jeremy's Comments: Find some examples; you have to justify why you talked about only a specific group of examples. Need to have a criteria for what is vulnerable to frontrunning: like top ten dapps from specific blogs. Have a criteria.

Blockchain technology has received an exceptional amount of attention since it was first introduced as the underlying technology of the cryptocurrency Bitcoin in Satoshi Nakamoto's (pseudonymous) 2008 white paper \cite{nakamoto2008bitcoin}. Many decentralized applications that are nowadays built on top of this technology represent the significance of the blockchain as it completely eliminates the single points of failures within any systems. However, blockchains have some inherent characteristic that leave them vulnerable to \emph{front running} behaviour. Although this data structure is known to be publicly visible to every network participants, information is layered  and some of them are accessed only by insiders. Any network participants that runs a full node on the blockchain is able to obtain those information and as mentioned in Section~\ref{sec:What is front-running?}, this leaves the application vulnerable to front running. Decentralized exchanges are a group of blockchain financial applications where front running can be executed. Bancor \cite{hertzog2017bancor} is an example of such systems in which front-runners benefit from potential price increase of the market stocks (details are provided in Section ...). Decentralized namespaces, as blockchain non-financial applications, are another example. Ghazal \cite{moosavighazal} is such system in which upon seeing a transaction from Alice to register \texttt{alice.com}, front runner can go ahead of her and register this domain name. He can later sell \texttt{alice.com} to Alice for a higher price and benefit from the price increase.
Another application that is vulnerable to front running is Initial Coin Offering, ICO, smart contracts. ICO is a method to distribute tokens based on the deposits of the native coin, Ether in the case of Ethereum. After a few hacking incidents on high valued smart contracts~\cite{siegel2016daohack}, ICOs started to implement restrictions and capped how much funds can be gathered. This scarcity of the initial coins made for a competition to incentivize big investors to get in and buy the tokens at a discounted price and sell them to late comers on the open markets~\cite{zetzsche2018ico, li2018initial}. ICOs started to experiment with different fair capping methods, such as reverse dutch auction, dynamic ceilings, \eg ~\cite{kaal2017initial}.  This competition made ICOs a good opportunity for front-running attacks.  In this paper we use multiple empirical examples to show possibility of front-running on DApps, for the sake of completeness, we use one incident on ICOs, and examples from the Top 25 DApps by recent user counts. %TODO: this last sentece is a bit weird here! 


%___________________________________________________________%

\section{Related Work}
As the traditional front-running was originated to trading financial instruments most of the literature are focused on financial aspects of the markets. %TODO: expand this with something


\subsection{Literature on Traditional Front-running}

% talk about the AMF frontrunning.

As traditional frontrunning has been the subject of discussion for decades, there are many literatures on this subject. Here we focus on the historical context and how the regulations evolved to be what they are today.\\
Front-running was first appeared on the Chicago Board Options Exchange (\textit{CBoE})~\cite{markham1988front}, and was identified by \textit{Securities Exchange Commission} in 1977 as the following:
\begin{quote}
The practice of effecting an options transaction based upon non-public information regarding an impending block transaction\footnote{Block in the stock market is large number of shares, 10\,000 or more, to sell which will heavily changes the price} in the underlying stock, in order to obtain a profit when the options market adjusts to the price at which the block trades. ~\cite{sec1978optionsmarket}
\end{quote} 

%In actuality, front-running is more complex than this definition suggests. It encompasses at least three forms of conduct, each of which raises different regulatory and policy issues.9 They are: (1) trading by third parties who are tipped on an impending block trade ("tippee" trading); (2) transactions in which the owner or purchaser of the block trade itself engages in the offset- ting futures or options transaction as a means of "hedging" against price fluctuations caused by the block transaction ("self-front-running"); 1 ° and (3) transactions where a broker with knowledge of an impending customer block order trades ahead of that order for the broker's own profit ("trading ahead").
% from JerryWMarkhamfront-running.pdf

% More to read and add from : SECURITIES AND EXCHANGE COMMISSION Release No. 34-67079. pdf included in Related_Documents
% https://www.sec.gov/about/annual_report/1988.pdf

On the years after there were ongoing discussions between self-regulatory exchanges (\eg \textit{CBoE}) and  \textit{SEC} to regulate, detect and define laws and regulations to deal with front-running~\cite{markham1988front}, with \textit{SEC} stating: 
\begin{quote}
It seems evident that such behaviour on the part of persons with knowledge of imminent transactions which will likely affect the price of the derivative security constitutes an unfair use of such knowledge. \footnote{Securities Exchange Act Release No. 14156, November 19, 1977, (Letter from George A. Fitzsimmons, Secretary, Securities and Exchange Commission to Joseph W. Sullivan, President  CBoE).}
\end{quote} 

As defining what exactly is considered illegal front-running required more knowledge of how these new markets behave, \textit{CBoE} and other exchanges (and brokers) issued educational circulars for their members asserting that front-running violates existing rules, with some examples of what is considered front-running. However difference of opinion regarding the unfairness o front running activities, insufficient exchange rules and lack of a precise definition in this area resulted in no action by self-regulator organizations~cite{sec1978optionsmarket}. 

Further reading on the early details on the history and challenges of detecting and regulating front-running can be found in~\cite{markham1988front} . % TODO: add a more recent publication here as well! 

Initially the front-running policies only applied to certain option markets, later on in 2002, the rule was refined to include the same prohibitions to security futures~\cite{finra_2002}, which then in 2012 with the new amendment, FINRA Rule 5270, the front-running rule was extended to cover trading in an option, derivative, or other financial instruments overlying a security with some exceptions~\cite{sec2012frontrunning, finra_2012}. 


\subsection{Literature on Blockchain Front-running}
%NOTE (ASK JEREMY): decentral or decentralized?
There are only a few research carried out on the front-running subject in the context of blockchain and decentral ledgers. \\

Aune~\etal discuss how the lack of time priority between broadcasting a transaction and its validation by miners on a blockchain based system would lead to the market information leakage~\cite{aune2017footprints}. As mentioned in Section~\ref{sec:What is front-running?}, the information leakage results the entire system to be vulnerable to the front-running. They also propose a cryptographic approach to prevent any decentralized system from front-running. In this approach, which is similar to commit and reveal, a transaction hash (fingerprint) is broadcast to the network to secure time priority while not revealing the information. Once the hash is processed and assigned an index and timestamp, the transaction details are broadcast. This makes the transaction and its hash to be executed at their reserved time priority and not vulnerable to front-running.

Malinova and Park discuss different design settings for financial markets and trading platforms on decentral ledgers ~\cite{malinova2017market}.  They argue that blockchain brings a new options of transparency which affects traders behaviour, specially in the presence of intermediary front-runners. They model indirect trading costs and use the economic literature on search and trading at decentralized exchanges in their design settings to make front-running costly and inefficient.


Breidenbach~\etal~\cite{breidenbach2018enter} face the front-running issue in the context of automating bug bounties for smart contracts. Upon submitting the bug bounty to the Hydra smart contract, one can front-run the bug report and claim the bounty instead of the original reporter. They propose \textit{Submarine Commitments} which is to conceal the bounty claim and the participatory factor. They claim submarine commitments ``are a stronger form of a commit-reveal scheme''. 

 
One of the properties of Blockchain technology was to overcome double-spend, however this does not apply for fast payments and zero-confirmation transactions. \textit{Double-spend attack}, as the name implies, the adversary spends a coin more than one time. To do so, he would use the coin as an input of a transaction and broadcast it to the network, before the transaction gets validated and executed, he broadcasts another transaction associated with a higher transaction fee which spends the same input ~\cite{bamert2013have, karame2012double}. This adversarial behaviour is similar to the front-running attack where the attacker would front-run other users by paying a higher transaction fee when broadcasting his transactions to the network, hence a higher priority transaction for miners to include in the block they mine.  


%______________________________________________________________%
\section{Front-running on Bancor}

Bancor is an Ethereum-based application that allows users to exchange their tokens without any counter-party risk. This protocol aims to solve the cryptocurrency liquidity issue by introducing \textit{Smart Tokens}~\cite{hertzog2017bancor}-- ERC20 compatible tokens with a built-in liquidity mechanism that are always available to users. Smart Tokens can be bought and sold through the users smart contract at an automatically calculated price which displays supply and demand. Doing so, Bancor protocol provides continuous liquidity for digital assets without relying on an order book as there is no requirement to match sellers and buyers.
Recently, researchers have shown that Bancor is vulnerable to front running attacks. Implemented on the Ethereum blockchain, when Bancor transactions are broadcast to the network, they sit in a pending transaction pool known as \textit{mempool} waiting for the miners to mine them. Since Bancor handles all the trades and exchanges on the Ethereum blockchain (unlike other existing decentralized exchanges), these transactions are all visible to the public for 16s (the average Ethereum blocktime) before being included within a block. This leaves this blockchain-based decentralized exchange vulnerable to the blockchain race condition attack as attackers are given enough time to front-run other transactions and gain favourable profits~\cite{BancorIs7:online}. As mentioned in Section~\ref{sec:who can front-run?} and same as every other Ethereum based application, Bancor transactions can be front run by miners and other regular (non-miner) users. There is an implementation of a Bancor front running attack in the Python programming language which represent how a non-miner user can front run other network participants \cite{NewTab13:online}.



\section{On Blockchain Front-running}

\subsection{Who Can Front-run?}
\label{sec:who can front-run?}

Blockchain, as a secure ledger, gives the impression of ultimate secure platform for decentral applications and one might not consider implementing the application with security in mind. In general, all the network participants have the ability to front-run specific transactions that are sent to the network. However, miners can include any transactions they like into the block they attempt to mine. Thus, they possess special power in terms of front-running as opposed to other users of the network. In the following, we discuss and compare the two groups of blockchain potential front-runners.

\subsubsection{Blockchain Participants}
Any regular (miner or non-miner) user can front-run other transactions in the network. For regular users to front-run others on the blockchain, they need to be well connected to other nodes on the network. Doing so, they are able to listen to the network and monitor all transactions that are broadcast. On the Ethereum blockchain, users have to pay for the computations in small amount of Ether called \textit{gas}~\cite{AccountT67:online}. The price that users pay for transactions (a.k.a. transaction fees) can increase or decrease how quickly miners will execute them and include them within the blocks they mine. This is because the Ethereum miners consume resources to process the transactions and so receive the transaction fees after creating the blocks. Once seeing two identical transactions with different transaction fees,  a rational miner will prioritize the transaction that pays higher gas price per unit of gas --gasPrice--, this is due to limited space in the blocks while maximizes their profit in the block they mine. Therefore, any regular users who run a full-node Ethereum client can front-run pending transactions by sending similar transaction with higher gas price, see Figure~\ref{fig:RegularFrontrunning}. For example, in a decentral exchange, he can monitor the network and once seeing a pending \textit{buy} transaction which will further increase the price of the asset, a front-runner can send a transaction with a higher gas price and hope to have his transaction mined ahead of the original pending transaction. By doing so, he achieves a better rate from the buyer. Note that in this case, blockchain front-runners are visible to the network as they broadcast their transactions to all the network participants. 

\begin{figure}[h]
\centering
\includegraphics[width=0.7\linewidth]{figures/Regular_frontrunning.png}
\caption{The Adversary up on seeing the Buy order, sends his own buy order with higher gas to front-run the first order. \label{fig:RegularFrontrunning}} 
\end{figure}
%https://www.lucidchart.com/invitations/accept/f8ca2662-5269-425d-9deb-deb14af2aac7

\subsubsection{Miners}
Blockchain miners are the only parties who can decide on the order of transactions within a block they mine, they can easily intercept and reorder the transactions in their blocks, this is known as \textbf{transaction reordering} attack.  For example in an Ethereum based application, a miner learns about the pending \textit{buy} transaction of 1000 units of a token, i.e. TKN, presumably if this transaction goes through, it causes the price of the purchased token to increase. So the dishonest miner can step in front of this transaction and place his own buy order ahead. He would simply create his \textit{buy} of 1000 TKN and include it in front of the previous \textit{buy} transaction of 1000 TKN in the block he mines. Doing so, he would receive a better rate than other network users, and afterwards can sell the assets he has bought and gain a price advantage at the expense of others. Similarly, a dishonest miner can sell his tokens in front, if he sees a pending \textit{sell} transaction. In an open source blockchain application, there is no rules on how transactions must be ordered and miners are free to mine transactions in any order they prefer. However, miners can only front-run transactions within the blocks they happen to mine. They could do this also by not broadcasting their own transactions to the network, this makes the miners to be less noticeable to the network when front-running.
Another example here is in the case of decentral exchanges and order cancellation. Every exchange should have the functionality to cancel the orders, especially for a volatile market. In this case, when user decides that her order is not profitable anymore, she would send a cancelation transaction. Here the miner sees the cancelation transaction and puts his own order in front of the cancellation transaction --transaction reordering-- to fill the order, potentially profiting from the order. They can also include the canceled transaction in their block to collect the full gas limit used by that transaction~\cite{CostofDecentralization:online}, see Figure~\ref{MinerFrontrunning}. Front-running can also occur in non-financial smart contracts. As an example, smart contract which adds all the participants to a party invite list, could only be closed by the smart contract owner. In this case when the owner sends the transaction to close the list, a miner can include his own list of participants before the close transaction in the block he mines. This is an issue with the design of the smart contracts and is known as transaction-ordering dependence vulnerability~\cite{luu2016making}. 

%TODO: figure for miners reordering attack. it should be just two blocks with transaction inside where the second "modified" one has the orders specifically to profit the miner.

\begin{figure}[h]
\centering
\includegraphics[width=0.7\linewidth]{figures/Miner_reorder_frontrun.png}
\caption{The adversarial miner can monitor the Ethereum mempool for decentral exchange cancel orders and upon seeing the cancellation transaction, he puts his buy order before ahead. Doing so, the miner can profit from the underlying trade and also get the Gas included in the cancel transaction. \label{fig:MinerFrontrunning}} 
\end{figure}
%https://www.lucidchart.com/invitations/accept/c8768786-4a3f-4c78-b3c8-aa04a6b81d06



\subsection{OCase Studies}  

As blockchain records are immutably recorded, there is enough historical data to analyze for possible front running detection. There is no perfect criteria to list top DApps on Ethereum blockchain, for the purpose of this research we analyzed the top 25 DApps based on recent user activities\footnote{List of decentralized applications \url{https://dappradar.com/dapps}}. 
%ICO is only in use when the time window is happenig.  this is based on the last 24hours activities 
%justify why started with ICO and go to other apps 


%TODO: fix the table!  %mahsa: add MISC back to the table! %%%% put a seperate cell for name services and put the ENS
% Mention that the categories were taken from that website (Dapp.com) OR recategorize categorize the table


% ------------------------------------------------------------------------------------------------------------------------------------------------%
% Top 25 Dapps Table:%

\begin{table}[t]
\centering
\begin{tabular}{|c|c|}
\hline
\textbf{DApp Category}    											& \textbf{DApp Names (Rank)} \\  \hline
\multirow{6}{*}{Exchanges} 										& IDEX (1) \\ 
															& ForkDelta, EtherDelta (2) \\
															& \textbf{Bancor (7)} \\
															& The Token Store(13) \\
															& LocalEthereum(14) \\
															& Kyber (22) \\  \hline

\multirow{9}{*}{\shortstack{Crypto-Collectible \\ Games \\ (ERC 721)}}		& \textbf{CryptoKitties (3)} \\  
															& Ethermon(4)\\
															& Cryptogirl (9)\\
															& Gods Unchained TCG (12)\\
															& Blockchain Cuties(15)\\
															& ETH.TOWN! (16)\\
															& 0xUniverse (18)\\
															& MLBCrypto Baseball (19)\\
															& HyperDragons(25)\\ \hline

					
\multirow{6}{*}{Gambling}											& Fomo3D (5) \\  
															& DailyDivs (6) \\ 
															& FomoWar (10) \\
															& FairDapp (11)\\
															& Zethr(17) \\
															& dice2.win (20) \\
															& Kyber (22) \\  \hline
												
\multirow{2}{*}{\shortstack{High-Risk \\ (Ponzi Schemes and Lotteries)}}		& PoWH 3D (8)	 \\ 
															& Ether Shrimp Farm(21) \\  \hline	

							
\multirow{2}{*}{Misc}												& 0x Protocol (23) \\
															& \textbf{Ethereum Name Service (24)}  \\  \hline	


\end{tabular}
\caption{\footnotesize{Top 25 DApps based on recent user activity from \texttt{Dappradar.com}.}\label{tab:top25dapps}}
\end{table}
% ------------------------------------------------------------------------------------------------------------------------------------------------%

From these DApps we looked at one example per category, most of which had front-running issues, some more severe than the other. Although none of these DApps are ICO smart contracts, we believe with the number of ICOs done on Ethereum it is worth to mention one instance of front-running on these applications as well. 


% maybe inlcude 0xbitcoin ERC721 mineable tokens front-running!
% mint: https://etherscan.io/tx/0x9883bc3ad018fd2b649982f88fbad7dc5abcb8f11c4f1d87ef814ba30c2b3428

\subsection{Status ICO} \hfill\\ 
\noindent As discussed in ~\ref{sec:Front Running on the Blockchains}, capped ICOs are a good application for front-running attacks.  In June 2017, Status.im~\cite{statuswhitepaper} started their ICO and within 3 hours they reached their cap, resulted in close to 300,000 Ether of funding. They used a \textit{fair} token distribution method called \textit{Dynamic ceiling}, which was in place to prevent large investments to take place early and limited the amount of Ether deposited, this was an attempt to increase the time window for smaller investors. They implemented multiple caps --ceiling-- which each would have a maximum amount that can be deposited. In this case every deposit would be checked by the smart contract and the exceeding amount would be refunded to the sender while the accepted amount would be sent to the Status.im wallet ~\cite{statusicoanalysis}. 
On the time of the ICO there were reports of Ethereum network being unusable and transactions were not confirming. Further study showed that some mining pools might have been manipulating the network for their own profit. On the other note, There were many transactions sent with higher gasPrice to front-run other transactions, however these transactions failed due to the restriction in place in the ICO smart contract to reject any transactions with higher than 50GWei gasPrice in an attempt to prevent this kind of front-running.


\begin{figure}[h]
\centering
\includegraphics[width=0.5\linewidth]{figures/Mining_pool_ratio.png}
\caption{The percentage of Ethereum blocks mined between block 3903900 and 3908029, this is the time frame in which Status.im ICO was running. This percentage roughly shows the hashing power ratio each miner had at that time. \label{fig:mining_pool_ratio}} % IS THIS STATEMENT CORRECT?
\end{figure}

F2Pool, an Ethereum mining pool that had around 23\% of the mining hash rate at the time (Figure ~\ref{fig:mining_pool_ratio}), sent 100 Ether to 30 new Ethereum addresses before Status.im ICO started. On the time the ICO started F2Pool constructed 31 transactions from the addresses they controlled to the ICO token sale smart contract --without broadcasting these transactions to the network-- and focused their mining power to mine their own transactions.
As shown in Figure~\ref{fig:Transactions_miners_while_status_ico_cut}, most of the top miners in the mentioned time frame, have mined almost the same number of failed and successful transactions which were directed toward Status.im token sale, however F2Pool's transactions indicate their successful transactions were equivalent to 10\% of the failed transactions, hence maximizing the mining rewards, on gas, while censoring other transactions to the token sale smart contract.

%TODO: talk about the tools made to figure this out , it's ok to use jargons and technical terms. maybe a paragraph or two on this, with interfaces and service. ethereum-etl, google bigquery, ...

\begin{figure}[h]
\centering
\includegraphics[width=0.7\linewidth]{figures/Transactions_miners_while_status_ico_cut_only_icotx.png}
\caption{This chart shows the miners behaviour on the time window that Status.im ICO was running. \textit{Successful Transactions} are the transactions that led to deposit and generation of Status.im tokens and \textit{Failed Transactions} are the transactions to the token sale smart contract that failed to retrieve any tokens, for varias reasons.  It is clear that the number of successful transactions mined by F2Pool do not follow the random homogeneous pattern of the rest of the network. \label{fig:Transactions_miners_while_status_ico_cut}} 
\end{figure}

%TODO: make pie charts for each mining pool, all non-status-ico transactions vs success-ico-transactions vs failed-ico=transactions

%TODO: describe failed/ successful transactions, what they mean! 



Involvement of F2Pool in this action becomes obvious when the transactions from these 30 addresses are traced. As shown in Figure~\ref{fig:f2poolfront-run}, the funds deposited by F2Pool in these addresses where sent to Status.im ICO and mined by F2Pool. Once the dynamic ceiling algorithm refunded a portion of the sent funds back to these addresses, these funds were again sent back to F2Pool main address. 


\begin{figure}[h]
\centering
\includegraphics[width=0.7\linewidth]{figures/F2Pool_transactions_to_StatusICO_and_Refunds.png}
\caption{F2Pool prior to Status.im ICO deposited 100 Ether in multiple new Ethereum addresses. On the time of the ICO they sent these deposits to Status ICO smart contract and prioritized mining of these transactions in their mining pool, this was to overcome the dynamic ceiling algorithm of the token sale smart contract. Later on they sent the refunded Ether back to their own address. [Graph made using Blockseer.com] \label{fig:f2poolfront-run}}
\end{figure}


Although this incident does not involve transaction reordering in the blocks, it shows how miners can modify their code to behave in certain way to front-run other transactions and result in monetary profit. The dataset used for this analysis can be found on our github repository\footnote{Frontrunning Status.im ICO dataset: \url{http://github.com/TODO:ANONYMOUS}}
%TODO: these might need more details and proofs. ask Jeremy about how to go on about this! 






%NOTE: @mahsa: is this considered a Historical evidence? should we talk about this somewhere else and maybe do a PoC on Ghazal for this section?
\subsubsection{Domain Name Front-running} \hfill\\    %Following section talks about front-running in namespaces but limited to Ghazal. Will talk about NameCoin in the section of mitigations (commit & reveal method)

%Jeremy's Comment: contrast ghazal vs ens, or try to have some historical evidence for namespace on the Ethereums. Try too make ghazal less central.
%Jeremy's comments: also talk about the "private information retrieval" (Ian Goldberg's work at waterloo)

\noindent As mentioned in Section~\ref{sec:Front Running on the Blockchains}, although front-running attacks have been more showcased in the context of decentralized exchanges and trading systems, they are are not yet limited to the financial markets. Front-running can occur within other (non-financial) blockchain applications such as naming systems. Blockchain-based domain registrar have been introduced to eliminate the role of central parties \ie domain name system (DNS) which introduce single point of failures in the entire web. One such system is Ghazal, an Ethereum-based naming and PKI system~\cite{moosavighazal}. Ghazal users rely on the Ethereum blockchain to register their \texttt{.ghazal} domain names and bind certificates to those names. In Ghazal model, a user would register domain name by executing the \textit{registerdomain} function from the Ghazal smart contract with the domain name in plain text as the function argument. As mentioned before, These transactions will sit in the mempool so that it would be mined by Ethereum miners and included in the block. During this period in which the transaction is not yet confirmed, front-running attack can occur by (i) dishonest miners and/or (ii) regular non-miner user. In the first case, a miner would intercept the \textit{registerdomain} transaction and register that name ahead of the user. A regular non-miner node in the Ethereum network can front-run other user's \textit{registerdomain} at a good profit by paying higher transaction fees. This practice allows the adversarial party  to register an unregistered domain immediately after someone searched for it, and sell the domain name to the users for a higher price. Note that, domain name front-running also occurs in the vanilla system. In this case, front-runners would monitor and identify domain names of interest, and sell it back to the other users with a higher price~\cite{sac022en33:online}.

%Some edits required here!

Ethereum Naming Service (ENS)\footnote{\url{https://ens.domains/}}, the most popular decentral naming service uses a sealed auction to purchase new domains with \textit{.eth} extension which can be resolved using any Ethereum web3 compatible browser. Although the implementation uses a sealed auction, \textit{startAuctionsAndBid}, the deposit should be equal or bigger than the bid, this information leakage could be used to front-run the transaction with the same bid amount. By doing so the front-runner invalidates the original bid. This is similar to traditional domain squatting where the service provider maliciously buys the domains upon domain availably check and before the user gets the chance to finish his order. Although the exact same attack is not feasible on blockchain naming services as querying already registered domains are instant and without leaking any information of the intention of the user to the network.
%e.g: https://etherscan.io/enslookup?q=handshaker.eth
%TODO: ian goberg: private information retrieval . talk about domain name frontrunning in the traditional sense and how goDaddy can steal the domain when queried for availablity. however in blockchain the queiry is private and does not reveal the intention that you want to buy the domain. View only / solidity code on how to enable this


% PoC of front-running ghazal

\subsection{Crypto-Collectibles auctions} \hfill\\ 
Crypto-Collectibles, also known as ERC-721 Non-Fungible Token Standard~\cite{erc721}, are one of the applications of Ethereum blockchain. Cryptokitties is known to be the most popular applications of ERC721 tokens~\cite{cryptokitties}. The main usage of these Cryptokitties is that users can breed them with each other and make new unique Cryptokitties, also they can buy or sell them through their main interface\footnote{Cryptokitties website \url{https://www.cryptokitties.co/}}. See Figure~\ref{cryptokittie}.
To buy a cryptokittie, user sends a bid transaction \textit{bid(uint256 \_tokenId)} containing the cryptokittties ID which can be found either on their website or on the ERC721 token smart contract. This is similar to open auctions and everyone can see the bid value and what is the bid for, this transaction is visible from any other Ethereum node and could be easily front-run by sending the same transaction with higher gasPrice. 
%TODO: this needs a rewrite. 
% https://etherscan.io/tx/0x62a76b7e345e8abd463487d0348c118a9314b197024e090fee7b1fbce434da40

\begin{figure}[h]
\centering
\includegraphics[width=0.4\linewidth]{figures/cryptokittie842912.png}
\caption{ Cryptokitty Number 842912 \label{fig:cryptokittie}}
\end{figure}


% Talk about how cryptokitties are similar to any other auctions . 

%______________________________________________________________%




\section{How to stop Front-running?} %We have have the applicability of the solutions inline.

In the traditional markets, front-running is considered unethical and also illegal. In the blockchain space, the designers of the decentral applications cannot rely on the justice system for unethical behaviour, and they must assume that each participant in the network acts rationally in their own self interest, the application will be operating in an adversarial environment~\cite{0xfrontrunning:online}.
A few decentral exchange projects such as \textit{EtherDelta} and \textit{0xProject}~\cite{warren20170x}, have proposed solutions for front-running, which is off-chain order books. The methods discussed within these projects prevent blockchain network participants from front running the orders as the orders are private and not known to the network and will not be broadcasted. However they are introducing third parties, \eg relayers in 0xproject, to be managing the orders with the promise not to reorder or front-run other orders. 

\emph{Traditional Front-running Prevention Methods}. regulations/enforcement/broker education/sealed order
%@shayan: ^


\emph{Blockchain front-running mitigation/prevention}. The traditional methods of preventing front-running are based on regulation and restrictions applied to the brokers and actors within the markets, such methods do not apply to blockchains, no enforcement methods, ... 

There are two main approaches to prevent front running, one to design a blockchain that is front-run resistant , and the other to design the application logics in a way that front running is not possible. 


\subsection{Front-run Resistant Blockchain}
What does this mean to have a front-run resistant blockchain?  There are technical difficulties to achieve such solutions as there are unknown factors within such network designs (corner/edge cases). In this section, we describe the potential solutions using which, one can design and implement a decentral application that is resistant to front-running.

\subsubsection{Design Decision \#1: Fixed Transaction Ordering\newline}

One possible solution to fix current blockchains in regards to transaction reordering by miners has been proposed by da Silva~\etal~\cite{fixedordering2018silva}. This solution consists of an algorithm, known as Fixed Transaction Ordering and Admission (FTOA) algorithm, in the consensus protocol that enforces the order of the transactions. 

\subsubsection{Design Decision \#2:Privacy Preserving Blockchain\newline}

%NEEDS TO BE DONE:
Market participants are concerned about revealing information on their past trades, purchases, income, or liquidity needs (forget about market participants >> this leads to frontrunning)
Shielded transactions in ZCash do not reveal the sender, receiver, the amount and the data included in the transactions, hence they cannot be seen by network participants to be front-run. Although this limits the functionality of the blockchain/smartcontracts/dapps 

\subsubsection{Design Decision \#3: Loopring/Dual Authoring\newline}
Talk about this possible solution

%TODO: research on more proposed solutions! 


\subsection{Application Design to Prevent Front-running}

Another method to prevent front-running activities is to design the applications, whether they are decentralized (Dapps) or centralized, in such a way that no information would be available to the malicious parties to enable them to front-run the other users.
Followings discuss the two techniques which can be used by the Apps/DApps developers while designing the systems to prevent front-running activities.

% TODO: also talk about drawbacks, more expensinve, more transactions, UX, forget to reveal.  



\subsubsection{Design Decision \#1: Commitment Scheme\newline} %talk about the drawbacks of this scheme: eg: you may forget to reveal. 
Commitment scheme is a cryptographic primitive that enables one to commit to a value --binding-- (\eg statement, document, data, \etc) while keeping it a secret and reveal the committed value on a later time~\cite{brassard1988minimum}. Commitment scheme is a robust method to prevent information leakage from the sensitive transactions. This could be done by simply broadcasting the hash value of the committed data and later on revealing the values that would be hashed to the committed hash, see Figure~\ref{fig:commitReveal}. 
In the case of decentral exchanges, user can send a commit transaction which will be cryptic to network participants but will act as a placeholder in the queue for the user, after the transaction is mined, user sends reveal transaction revealing the order details which will be executed in a fair order. Depending on the Dapp design, either the order is collateralized, which means the commitment transaction includes the funds required for fulfilling the order, which will leak some information about the order, or is not collateralized and opens up the possibility of user never revealing the commitment, which will be considered as a DoS attack. 
Another use of such commitment scheme, is in decentralized naming systems, such as Ghazal or ENS. In this case, a user would send a commit transaction similar to a sealed bid, once the transaction is confirmed and the grace period is over, user would send a reveal transaction revealing the bid and also the details of the requested domain. Using this scheme, one is able to hide information from the adversarial parties in the system and prevent them from front-running. Similar approach applies to on-chain voting as well ~\cite{adChainPLCRVoting}, using sealed votes, voters can make sure their votes are hidden until a later date, however their participation and weight of the vote are publicly known. 
Other than the collateralization issue for the commit and reveal scheme, another obvious issue is the participatory factor. For anyone watching the DApp address, there will be a direct transaction from the user to the DApp address, It will be obvious that a specific address has participated in the auction or the DApp using a commit and reveal scheme, but the details are hidden through the scheme. Both these issues, albeit hiding the details of the order and preventing direct front-running, leak information to other participants which could lead to more sophisticated front-running attacks depending on the application design.

\begin{figure}[h]
\centering
\includegraphics[width=0.5\linewidth]{figures/commit-reveal.png}
\caption{ Commit and Reveal. User sends a commitment transaction with the hash of the data, After the commitment period is over, user sends her reveal transaction to the DApp revealing the information that matches the commitment. \label{fig:commitReveal}}
\end{figure}
%https://www.lucidchart.com/invitations/accept/3cd4c865-83d4-4c71-9a4a-f2e870f2db2c

\subsubsection{Design Decision \#2: Submarine Commitments\newline}
Submarine Commitments~\cite{breidenbach2018enter, submarinesendHD} are similar to the commitment schemes with some key differences such as hiding participatory factor and being fully collateralized. However, they could be design in other ways to suit the need of the DApp. With submarine send, it is possible to hide sender, receiver, value and data, the commitment transaction is identical to a transaction to a newly generated address. Submarine sends could be a solution for sealed-bid auctions on the blockchain to hide the existence of the bids for other participants. 

The details of how Submarine commitments work is outside the scope of this paper, however in summary, using the way Ethereum transactions are constructed and ECDSA ECRecover functionality it is possible to generate one outgoing transaction from an address in which no private key exists. By constructing a transaction from that address to the DApp and funding that address we fulfil the commitment phase and by revealing the details and broadcasting the constructed transaction we reveal the commitment ~\cite{libsubmarine}, see Figure~\ref{fig:LibSubmarine}. 


\begin{figure}[h]
\centering
\includegraphics[width=0.5\linewidth]{figures/LibSubmarine.png}
\caption{ Submarine Send~\cite{libsubmarine}. User generates an \textit{Unlock} transaction from which the commitment address is retrieved. By funding the \textit{commitment address}, user is committed to the unlock transaction. After the commitment transaction, in the reveal phase, user sends the \textit{reveal transaction} to the DApp and then after she can broadcast the Unlock transaction to unlock the funds in the commitment address. Finally after the "Auction'' is over, anyone can call \textit{Finalize} function to finalize the process.  \label{fig:LibSubmarine}}
\end{figure}
%https://www.lucidchart.com/invitations/accept/3a259b74-1162-4fb5-8023-eb0311018295


\subsubsection{Design Decision \#3: Logic Specific Solutions\newline}
To prevent front-running, one can design the application in a way that it is not possible to be front-run:

\noindent\emph{Eliminating Time-Order Dependency:} The applications can be designed to not rely on the time for the orders to be executed. For example when designing a decentral exchange, one can use \emph{the call market design} instead of a time-sensitive order book. In a call market design, the arrival time of the orders do not matter as they are executed in batches [cite Jeremy's prediction markets paper]. 

\noindent\emph{Disincentivizing Front-running Actors:} This solution, which is economic rather than technical, is to disincentives the miners by paying them. As mentioned, the main reasons for miners to front-run the other orders is the financial gain they receive on the price improvement. However, if the application is designed in a way that the fees of the orders are sent to the miner of the block, he would have no incentive to front-run the orders as he already gains enough financial benefit from acting fairly and executing the transactions on their correct order.

%TODO: maybe remove this 
%\noindent\emph{Eliminating Price Dynamics:} In the context of decentralized exchanges where orderly and fair execution of the orders (transactions) cannot be guaranteed,  another solution is to replace the order book with the \emph{limit order book}. In such design, orders are not executed automatically and an auction process helps the users to choose the orders they want to execute against. By doing so, even if the orders are front-run by other users or miners, there would be no price improvement and financial benefits as the orders are limited. 

Note that depending on the applications plans, there might exist design decisions which cause the system to be front-run resistant. For example, Malinova~\cite{Malinova} proposes a design for a decentral exchange which uses economic models in their trading market design to make front-running expensive and more costly than profitable.

Going back to Table~\ref{tab:top25dapps}, some DApps facilitate different methods to mitigate front-running issues. As an example LocalEthereum, a escrow-based exchange, uses challenge random values from the website to be included in the signature of the transaction (See ~\ref{code:localethereum}), hence removing any possibility of replay attack or possibility of any other address sending the similar order.

\begin{lstlisting}[basicstyle=\scriptsize\ttfamily,caption={Code snippet from LocalEthereum smart contract. Values V,R and S are set by LocalEtherem to have a valid signature, also the tradeHash uses buyer and seller addresses, mitigating the possibility of front-running by a third party.},label={code:localethereum},float]
    function createEscrow(bytes16 _tradeID, address _seller, address _buyer, uint256 _value, uint16 _fee,
					uint32 _paymentWindowInSeconds, uint32 _expiry, uint8 _v, bytes32 _r, bytes32 _s) 
        payable external {
        bytes32 _tradeHash = keccak256(abi.encodePacked(_tradeID, _seller, _buyer, _value, _fee));
	...
        // A signature (v, r and s) must come from localethereum to open an escrow
        bytes32 _invitationHash = keccak256(abi.encodePacked(
            _tradeHash,
            _paymentWindowInSeconds,
            _expiry
        )); 
\end{lstlisting}

%Proof of burn methods ? 
%TODO: research on more possible solutions! 


%TODO: inline talk about what these solutions are good for and if they fit Status ICO


%______________________________________________________________%




\section{Concluding Remarks}



% Could talk about a tool / a framework to detect front-running


%\subsubsection*{Acknowledgements.}
